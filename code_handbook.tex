\documentclass[12pt,a4paper]{article}
\usepackage{amsmath}
\usepackage{amsthm}
\usepackage{amsfonts}
\usepackage{amssymb}
\usepackage{amsmath,amscd}
\usepackage[symbol]{footmisc}
\usepackage{fancyhdr}
\usepackage{graphicx}
\usepackage{bm}
\usepackage[english]{babel}
\linespread{1.25}


\DeclareMathOperator{\Tr}{Tr}
\DeclareMathOperator{\D}{D}
\DeclareMathOperator{\Real}{Re}
\DeclareMathOperator{\T}{\text{\scriptsize T}}

\renewcommand{\thefootnote}{\fnsymbol{footnote}}


\begin{document}


\section{Powers of $D$}
For many applications it is useful to write $\Tr D^p$ in a meaningful form.\newline
$\Tr D^p$ expands to:
\begin{align}\label{eq:trdp}
\Tr D^p &= \sum_{i_1, \ldots, i_p} \Tr (\omega_{i_1} \cdots \omega_{i_p}) \cdot \notag \\
&\Tr \big( (M_{i_1} \otimes I + \epsilon_{i_1} I \otimes M_{i_1}^T) \cdots  (M_{i_p} \otimes I + \epsilon_{i_p} I \otimes M_{i_p}^T) \big).
\end{align}
Focus first on the product over (anti-)commutators:
\begin{equation}\label{eq:prod}
(M_{i_1} \otimes I + \epsilon_{i_1} I \otimes M_{i_1}^T) \cdots  (M_{i_p} \otimes I + \epsilon_{i_p} I \otimes M_{i_p}^T).
\end{equation}
The following observations are useful for writing the product explicitly.
\begin{enumerate}
	\item The terms appearing in the product are given by all possible ways to distribute $r$ matrices ($0 \le r \le p$) on the left side of the tensor product, and the remaining $s = p-r$ matrices on the right, but keeping the order of the indices (on each side separately). This means that for example:
$$ M_{i_1} M_{i_p} \otimes M_{i_2}^T \cdots M_{i_{p-1}}^T $$
is a valid term with $r=2$ and $s=p-2$, but:
$$ M_{i_p} M_{i_1} \otimes M_{i_2}^T \cdots M_{i_{p-1}}^T $$
is not, because on the left side the index $i_p$ appears before $i_1$. When $r=0$ (respectively $s=0$) it is intended that there is an identity matrix on the left (right) side of the tensor product.
	\item Each term is multiplied by a product of $\epsilon_i$ factors given by the matrices that appear on the right. In the case of the previous example:
$$ \epsilon_{i_2} \cdots \epsilon_{i_{p-1}} M_{i_1} M_{i_p} \otimes M_{i_2}^T \cdots M_{i_{p-1}}^T. $$
	\item For each term with $r$ matrices on the left, there is a corresponding one with the same $r$ matrices on the right. Again using the same example:
$$ \epsilon_{i_1} \epsilon_{i_p} M_{i_2} \cdots M_{i_{p-1}} \otimes M_{i_1}^T M_{i_p}^T. $$
\end{enumerate}
It follows that the product (\ref{eq:prod}) can be written as:
\begin{align}\label{eq:prodsum}
\sum_{r=0}^{\left \lfloor \frac{p}{2} \right \rfloor} \ \ \sum_{j_1 < \ldots < j_r = 1}^p \Big[ &\epsilon_{i_{k_1}} \cdots \epsilon_{i_{k_s}} (M_{i_{j_1}} \cdots M_{i_{j_r}}) \otimes ( M_{i_{k_1}}^T \cdots M_{i_{k_s}}^T ) + \notag \\
+ &\epsilon_{i_{j_1}} \cdots \epsilon_{i_{j_r}} ( M_{i_{k_1}} \cdots M_{i_{k_s}} ) \otimes ( M_{i_{j_1}}^T \cdots M_{i_{j_r}}^T ) \Big]
\end{align}
where $\lfloor \ \rfloor$ denotes the floor function (i.e., the sum in $r$ runs from 0 to the greatest integer less than or equal to $p/2$), $j_1, \ldots, j_r$ are a set of $r$ indices picked from $\{ 1, \ldots, p \}$ in increasing order, and $k_1, \ldots, k_s$ are the remaining ones, also in increasing order.\newline
Recall now that $\epsilon_i^{-1} = \epsilon_i$ and that $M_i^\dagger = M_i$ for all $i$, so in particular $M_i^T = M_i^*$. Tracing over the product then gives:
\begin{equation}
\sum_{r=0}^{\left \lfloor \frac{p}{2} \right \rfloor} \ \ \sum_{j_1 < \ldots < j_r = 1}^p \epsilon_{i_{j_1}} \cdots \epsilon_{i_{j_r}} [ 1 + \epsilon^p *] \Tr(M_{i_{j_1}} \cdots M_{i_{j_r}})^* \Tr( M_{i_{k_1}} \cdots M_{i_{k_s}} )
\end{equation}
where $\epsilon^p$ is defined to be the product $\epsilon_{i_1} \cdots \epsilon_{i_p}$ and the in-line operator $*$ means complex conjugation of everything that appears on the right.\newline
Notice that the generic term in the sum only depends on the choice of $j_1, \ldots, j_r$. Denote the geneirc term $T(j_1, \ldots, j_r)$. The formula for $\Tr D^p$ can finally be written coincisely as:
\begin{equation}
\Tr D^p = \sum_{i_1, \ldots, i_p} \Tr (\omega_{i_1} \cdots \omega_{i_p}) \Bigg[  \sum_{r=0}^{\left \lfloor \frac{p}{2} \right \rfloor} \ \ \sum_{j_1 < \ldots < j_r = 1}^p T(j_1, \ldots, j_r) \Bigg]
\end{equation}


\end{document}
